El diputado Juan José Huanca Mamani presentó el proyecto de ley fechado el 1 de marzo de 2023, titulado ``Proyecto de Ley que regula y sanciona el uso indebido de las redes sociales en todo el territorio del Estado Plurinacional de Bolivia''. Esta propuesta detalla disposiciones legales específicas para regular el uso problemático de las redes sociales en el país. Aunque el proyecto no se enfoca de manera individual en el lenguaje ofensivo en las plataformas, aborda principios generales fundamentales.

En el artículo 3 sobre principios generales, se resalta la importancia del respeto hacia terceras personas en el entorno de las redes sociales: ``En el uso de las redes sociales en sus diferentes ámbitos, se debe tener presente el respeto hacia las terceras personas en un espacio amplio de navegación vía internet, donde cada usuario tiene su propia opinión, que no siempre se debe compartir, pero sí respetar'' \cite[Articulo 3, PL. 304 2022-2023]{diputados2023ley}.

El artículo 6 establece restricciones claras, prohibiendo la publicación de información, comentarios, imágenes o videos ofensivos, amenazantes o que afecten la imagen personal, la honra, la intimidad, la integridad personal o la libertad de expresión en internet. También prohíbe el uso de un lenguaje violento que incite al odio o la discriminación en formas prohibidas por la ley \cite[Articulo 6, PL. 304 2022-2023]{diputados2023ley}.

Los artículos 8 y 9 establecen las sanciones para quienes atenten contra la dignidad y el honor de otros a través de las redes sociales, con penas que pueden llegar a una privación de la libertad de entre cinco y siete años.
Aunque el proyecto no define términos clave como ``ofensivo'' o ``violento'', se evidencia la intención de no tolerar conductas que involucren este tipo de comportamientos en las redes sociales \cite[Articulo 3, PL. 304 2022-2023]{diputados2023ley}.
