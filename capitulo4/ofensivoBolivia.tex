En esta sección, se explorará en detalle las características distintivas del lenguaje ofensivo en las plataformas digitales bolivianas. Se analizará sus manifestaciones, impacto y las particularidades  en los enfoques lingüísticos ya explicados anteriormente. Se consideró de suma importancia proporcionar ejemplos claros para ilustrar lo que cada enfoque lingüístico pretende explicar. Cada frase, término o palabra utilizada como ejemplo en esta sección se extrajo de un conjunto de datos compuesto por comentarios recolectados tanto de la red social Facebook como de WhatsApp.

\begin{itemize}

	\item Sociolingüística del lenguaje ofensivo en Bolivia:  Se ha logrado notar un panorama complejo y diverso en el entorno digital y social del país de Bolivia. En las redes sociales de Bolivia, un país con una rica mezcla cultural, conformada por comunidades como aymaras, quechuas, urus, chiquitanos, guaraníes y otras, se ha identificado lenguaje ofensivo que tiende a jerarquizarse y dirigirse hacia grupos étnicos. Por ejemplo, el término ``chola(o)'', inicialmente utilizado para describir a mestizos, se ha degradado en un insulto asociado con comportamientos considerados vulgares. Frases como ``Gritas como chola del mercado'' o ``Ha estudiado en la universidad pese a que solo es un cholito'' son ejemplos palpables de esta dinámica. Otros términos como ``indio'', ``cunumi'', ``chota'', entre otros, también se emplean con connotaciones despectivas.
	
En el contexto religioso, donde el cristianismo, principalmente la Iglesia Católica, ejerce una fuerte influencia en la cultura boliviana, se han registrado manifestaciones de intolerancia hacia la comunidad LGBTQ+.Esta actitud también se refleja en numerosos comentarios en redes sociales como Facebook y en grupos de chat de WhatsApp, donde se utilizan términos peyorativos como ``gay'', ``maricón'', ``desviado'' y otros similares. El uso de este lenguaje contribuye a perpetuar estereotipos negativos y fomenta la discriminación contra estos grupos.

Asimismo, en ambientes menos formales como algunos chats de grupos sociales, se observa una frecuente presencia de palabras malsonantes, racistas y homofóbicas en la comunicación cotidiana. Términos como ``perro'', ``mierda'', ``gil'' y otros similares se han integrado de manera habitual en dichos entornos, reflejando el uso coloquial de un lenguaje con marcadas connotaciones ofensivas.

	\item Pragmática del lenguaje ofensivo en Bolivia .- Al igual que en otras sociedades, en Bolivia se ha arraigado el uso de lenguaje ofensivo indirecto, empleando expresiones coloquiales que transmiten críticas o desprecio sin recurrir a términos vulgares. Ejemplos comunes de estas frases son utilizados con frecuencia en redes sociales y chats. Por ejemplo, la expresión ``con suerte saliste del colegio'' se utiliza de manera sarcástica o irónica insinuando que alguien posee una educación limitada o escasa formación. Del mismo modo, frases como ``tu casa tiene techo de calamina'' se usan para referirse a alguien como perteneciente a un estrato socioeconómico de escasos recursos, mostrando así la sutilidad del lenguaje ofensivo presente en el entorno social y digital.
	
	\item Dialectología en el lenguaje ofensivo en Bolivia.- En Bolivia, el español es el idioma más hablado y se encuentra acompañado por el quechua, el aimara y otras lenguas indígenas, reconocidas como cooficiales en sus respectivas regiones. En el español boliviano de las áreas orientales, como Santa Cruz, se han detectado variaciones sintácticas y fonéticas que difieren del uso común del español. Por ejemplo, expresiones como ``voj'', ``somoj'' y ``tenej'' (cuyas escrituras correctas son ``vos'', ``somos'' y ``tenes'' respectivamente) son habituales en las redes sociales asociadas con estas zonas. Estas variaciones también se extienden a palabras malsonantes utilizadas para ofender, como ``webadaj'', que equivale a ``webadas'' y es una expresión coloquial empleada en algunos contextos para referirse a tonterías de manera más abrupta y vulgar o ``cagaj'' que es un término equivalente a ``cagas'' y se utiliza como expresión vulgar para referirse a errores o fallos.
	
	\item Semántica en el lenguaje ofensivo en Bolivia.- En las redes sociales bolivianas, ciertas palabras y frases que en un principio no eran consideradas ofensivas han adquirido connotaciones negativas debido a su uso despectivo. Términos como ``masista'', ``arcista'' o ``pitita'', originalmente referidos a afiliaciones políticas específicas, han evolucionado hacia juicios sobre el nivel educativo y clases sociales. Esto es notable en expresiones como ``no sabe ni hablar, seguro es masista'', ``es narco, es amigo de los masistas'' o ``no puede ni cortarse las uñas, se nota que es un pitita'', son frases comunes en redes sociales.

	\item Sintaxis en el lenguaje ofensivo en Bolivia.- En el panorama de las redes sociales bolivianas, se han hallado numerosas palabras ``nuevas'' creadas exclusivamente con el propósito de ofender. Estas combinaciones, muchas dirigidas a preferencias políticas, se han construido de manera numerosa, estas pueden ser: ``masibestia'', ``masillama'', ``masiburro'', ``masirata'', ``pitirata'', ``narcoburro'' y similares. Estas fusionan denominativos de grupos políticos con nombres de especies de animales usados de manera despectiva o adjetivos insultantes. Asimismo, se han identificado insultos habituales pero con caracteres omitidos, que al autocompletarse resultan ofensivos: ``crjo'' como abreviatura de carajo, que expresa enfado, sorpresa o molestia, ``hdp'' para ``hijo de puta'', una ofensa directa que alude a la madre de alguien como prostituta, y ``cjdo'' como representación de cojudo, un término peyorativo para referirse a alguien como tonto.
	
	\item Morfología en el lenguaje ofensivo en Bolivia.-En el contexto boliviano, se han identificado en numerosos comentarios de redes sociales como Facebook y WhatsApp, que hacen el uso de términos despectivos y ofensivos que alteran el significado de palabras comunes para desacreditar o menospreciar, tales como ``oficialucho'', ``medicucho'', ``marimacho''.Además, se recurre a la reduplicación para intensificar el tono ofensivo de los mensajes, con ejemplos como, ``ladroooooon'', ``naaaarco'', ``cooooorrupto'', ``caraaaajo'', entre otros.
\end{itemize}