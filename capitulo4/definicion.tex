En la actualidad, se encuentran bastantes diferencias terminológicas para referirse a lo que se puede considerar lenguaje ofensivo, si bien ofensivo, es el término que se decidió adoptar para fines de este proyecto, en distintos estudios lingüísticos, se ha abordado este tipo de lenguaje bajo otras denominaciones como ``lenguaje sucio'', ``lenguaje fuerte'', ``lenguaje soez'', ``lenguaje tabú'', ``lenguaje grosero'', ``lenguaje cargado de emociones'', entre otros.
Se presentan dos definiciones para ilustrar la variedad de enfoques sobre este tipo de lenguaje:

``Cualquier palabra o cadena de palabras que tiene o puede tener un impacto negativo en el sentido de sí mismo y/o el bienestar de aquellos que se encuentran con ella, es decir, hace o puede hacer que se sientan leve o extremadamente desconcertados y/o insultados y/ o heridos y/o asustados'' \cite[p. 16]{o2020offensive}.


``El lenguaje ofensivo se refiere a aquellos términos lingüísticos o expresiones compuestas por palabrotas, maldiciones, etc., que normalmente se consideran despectivos y/o insultantes.''\cite[p. 28]{avila2016treatment}.

Estas definiciones representan solo un fragmento de las diversas interpretaciones que existen. Es esencial notar que, independientemente del enfoque adoptado para definir este tipo de lenguaje, todos resaltan su característica principal: su negatividad general percibida por aquellos que lo leen o escuchan.


