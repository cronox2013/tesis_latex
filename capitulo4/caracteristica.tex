Las características del lenguaje abarcan un amplio espectro de elementos que definen la naturaleza y la función de la comunicación humana. La lingüística es la disciplina dedicada al estudio científico del lenguaje. Se enfoca en entender cómo se estructuran, funcionan y se utilizan los distintos sistemas de comunicación humana, ya sea oral, escrito o gestual. La lingüística abarca diversas áreas de estudio, como la morfología, la sintaxis, la semántica, entre otras importantes ramas.  Es por esa razón que se detallara a continuación el área de estudio de cada una y su relación con el lenguaje ofensivo:

Sociolingüística del lenguaje ofensivo. La sociolingüística es una rama de la lingüística que se centra en el estudio de la relación entre el lenguaje y la sociedad. ``La linguistica y la sociologia establecen relaciones entre los usos y actitudes linguisticas de una sociedad donde coexisten lenguas, razas y culturas completamente complejas y en el que las personas dependen unas de otras reciprocamente presuponiendo relaciones y conflictos. Se considera a la lengua y a la sociedad unidos, ambos, no pueden ser considerados aisladamente al margen de la ``cultura''.'' \cite{prado2004analisis}.

El uso de lenguaje ofensivo está influenciado por factores sociales, como la cultura, la identidad, la jerarquía social y las actitudes hacia ciertos grupos o individuos. Aquí hay algunas características sociolingüísticas del lenguaje ofensivo:

\begin{itemize}
		\item Variación según contextos sociales.- El lenguaje ofensivo puede variar según el contexto social y cultural en el que se utilice. Las palabras o expresiones que se consideran ofensivas en un entorno pueden no serlo en otro, dependiendo de las normas y las actitudes lingüísticas de esa comunidad.
		\item Jerarquía y poder.-  El lenguaje ofensivo a menudo refleja relaciones de poder y jerarquía. Puede utilizarse para denigrar a grupos o individuos que se perciben como menos poderosos o marginados en una sociedad.
		\item Efectos en la percepción social.- El uso de lenguaje ofensivo puede influir en la percepción social y en cómo se ven y se tratan ciertos grupos. Puede reforzar estereotipos negativos y contribuir a la discriminación. 
		\item Cambios en las actitudes lingüísticas.- Las actitudes hacia el lenguaje ofensivo pueden cambiar con el tiempo debido a cambios culturales y sociales. Lo que se consideraba aceptable en el pasado puede ser considerado inaceptable en el presente o igualmente de forma contraria.
		\item Códigos y contextos informales.- El lenguaje ofensivo a menudo se encuentra en contextos informales, como conversaciones entre amigos o en interacciones en línea. Esto puede reflejar la relajación de las normas lingüísticas en contextos menos formales.

\end{itemize}

Pragmática del lenguaje ofensivo. La pragmática es un campo de estudio lingüístico que se enfoca en el significado del lenguaje en contexto, es decir va más allá de la estructura formal de las palabras y oraciones. Como expresa \citeA{dascal1999pragmatica} La pragmática es el estudio del uso de los medios lingüísticos a través de los cuales un hablante dirige sus intenciones comunicativas y un oyente las reconoce.
	
La pragmática en el lenguaje ofensivo implica analizar cómo se interpreta las expresiones ofensivas en contextos de comunicación. Considerar los aspectos pragmáticos es esencial para comprender por qué y cómo se emplea el lenguaje ofensivo en diferentes situaciones y cómo afecta a la interacción comunicativa. Algunos aspectos pragmáticos relacionados con el lenguaje ofensivo incluyen:

\begin{itemize}
		\item Implicaturas y significado indirecto.- El lenguaje ofensivo a menudo utiliza implicaturas conversacionales, donde el significado implícito de una expresión va más allá del significado literal. Los hablantes pueden emplear el lenguaje ofensivo de manera indirecta para transmitir su intención ofensiva sin usar un lenguaje explícitamente vulgar.
		
		\item Contexto y relación.- La interpretación del lenguaje ofensivo depende del contexto de la comunicación y de la relación entre los interlocutores. Lo que puede ser ofensivo para un oyente puede no serlo para otro, dependiendo de la familiaridad y la percepción mutua entre las partes.
		
		\item Intención y efecto.- La interpretación del lenguaje ofensivo a menudo involucra analizar tanto la intención del hablante como el efecto que el mensaje tiene en el receptor. La percepción de la intención puede variar, y el lenguaje ofensivo puede tener diferentes efectos emocionales y psicológicos en diferentes personas.
		
		\item Atenuación y refuerzo.- Los hablantes pueden utilizar estrategias pragmáticas para atenuar o reforzar el impacto del lenguaje ofensivo. Por ejemplo, el uso de sarcasmo puede atenuar el impacto negativo o, por el contrario, reforzar la ofensa al comunicar una actitud despectiva.
\end{itemize}

Dialectología en el lenguaje ofensivo. La dialectología es una rama de la lingüística que se ocupa del estudio de los dialectos. Los dialectos son variedades de una lengua que difieren en aspectos fonéticos, gramaticales, léxicos y hasta pragmáticos, y pueden variar de una región a otra, dentro de un país o incluso entre países. ``La Dialectologia, rama de la linguistica, se ocupa de estudiar la variacion linguistica dentro de una sociedad, el habla de cada dia que no tiene cultivo literario y que ademas es propio de cada individuo, tomando en cuenta su diferenciacion o realidad que puede escaparse del esquema de cualquier norma y producir el cambio linguistico tanto en forma como en significado'' \cite[p. 20]{prado2004analisis}.

La dialectología en el contexto del lenguaje ofensivo se refiere al estudio de cómo las variaciones dialectales influyen en la forma en que se emplea y se interpreta el lenguaje ofensivo en diferentes regiones o comunidades lingüísticas. Aquí hay algunas consideraciones específicas de la dialectología en relación con el lenguaje ofensivo:

\begin{itemize}
			\item Tabúes y eufemismos: Las variaciones dialectales también pueden influir en cómo se tratan los tabúes lingüísticos y el uso de eufemismos. Algunas palabras ofensivas pueden ser reemplazadas por términos menos ofensivos en ciertos dialectos, mientras que en otros dialectos pueden mantenerse sin censura.
			
			\item Influencia en la intensidad: Algunos dialectos pueden utilizar palabras o estructuras gramaticales que intensifican el lenguaje ofensivo, mientras que otros pueden ser más sutiles en su uso. Esto puede afectar la percepción del nivel de ofensa.

\end{itemize}

Semántica en el lenguaje ofensivo. La semántica se enfoca en el estudio del significado en el lenguaje, es decir, cómo las palabras, frases, oraciones y discursos comunican significado. Para \citeA[p.28]{dascal1999pragmatica}, la semántica se ocupa de la determinacion del significado de la oracion (cuyo objeto es la descripcion y analisis de los significados convencionalizados y reglamentados) independientemente de su uso, y tambien del 'significado de la exclamacion' teniendo en cuenta aquellos aspectos del contexto de uso previstos por la estructura semantica de la oracion expresada.

``La semantica tiene por objeto el estudio del significado de los signos linguisticos'' \cite[p. 36]{prado2004analisis} .

La semántica en el lenguaje ofensivo se refiere al estudio del significado de las palabras y expresiones ofensivas, así como a, cómo se construye y se interpreta el significado en este contexto. Aquí hay algunas consideraciones específicas de la semántica en el lenguaje ofensivo:

\begin{itemize}
			\item Cambios de significado: Algunas palabras en el lenguaje ofensivo pueden tener significados diferentes de los que tienen en otros contextos. Estos cambios semánticos pueden ser intencionales y se utilizan para transmitir desprecio, desaprobación o insulto.
			
			\item	Expresiones idiomáticas ofensivas: Las expresiones idiomáticas también pueden tener connotaciones ofensivas. Estas expresiones pueden ser difíciles de interpretar para quienes no están familiarizados con la cultura y el contexto en el que se usan.
			
			\item Relación entre palabras: La semántica en el lenguaje ofensivo también se relaciona con cómo las palabras ofensivas interactúan entre sí en una expresión o una oración, creando un mensaje ofensivo completo.
			
			\item Neologismos ofensivos: La creación de nuevas palabras ofensivas o la asignación de nuevos significados ofensivos a palabras existentes también es parte de la semántica en el lenguaje ofensivo.
			
\end{itemize}

Sintaxis en el lenguaje ofensivo.-  Esta disciplina se concentra en cómo las palabras se combinan y se organizan para crear significados dentro de un idioma.``Parte de la gramática que estudia el modo en que se combinan las palabras y los grupos que éstas forman para expresar significados, así como las relaciones que se establecen entre todas esas unidades'' \cite[``Sintaxis'' definicion 1]{rae2023Online}.

La sintaxis en el lenguaje ofensivo se refiere al estudio de cómo se estructuran las oraciones y las expresiones ofensivas, así como a, cómo las reglas gramaticales pueden influir en la forma en que se construyen mensajes ofensivos.Aquí hay algunas consideraciones específicas de la sintaxis en el lenguaje ofensivo:
		
\begin{itemize}
		\item Orden de las palabras.- La sintaxis puede ser utilizada estratégicamente para enfatizar ciertas partes de un mensaje ofensivo. Cambiar el orden de las palabras puede acentuar el impacto negativo de ciertas expresiones.
		
		\item Uso de adjetivos y adverbios.- La elección de adjetivos y adverbios puede tener un impacto en cómo se percibe un mensaje ofensivo. Estos elementos pueden amplificar la connotación negativa o destacar ciertos aspectos que se quieren enfatizar.
		
		\item Comparaciones y metáforas: La sintaxis del lenguaje ofensivo puede involucrar comparaciones y metáforas que magnifican la ofensa. Estas estructuras pueden aumentar la intensidad del mensaje.
		
		\item Elipsis y omisión de elementos: La sintaxis puede implicar la omisión de elementos en una oración, lo que puede requerir que el receptor complete el significado de manera ofensiva.
		
		\item Inversión y acentuación: La inversión de estructuras gramaticales o la acentuación de ciertos elementos pueden hacer que un mensaje ofensivo sea más impactante y llamativo.
		
		\item Usos especiales de conectores: El uso de conectores puede dar un tono sarcástico, irónico o despectivo a un mensaje ofensivo.
		
		\item Estructuras retóricas: La sintaxis del lenguaje ofensivo puede involucrar el uso de preguntas retóricas o exclamaciones para intensificar el mensaje.
		
		\item Uso de subordinación: Las oraciones subordinadas pueden ser utilizadas para agregar detalles ofensivos o para enfocar la atención en ciertos aspectos que se quieren destacar.
		
\end{itemize}

Morfología en el lenguaje ofensivo.- La morfología se encarga del estudio de la estructura, formación y clasificación de las palabras. Así como de sus componentes más pequeños y significativos, llamados morfemas. ``Parte de la gramática que estudia la estructura interna de las palabras y de sus elementos constitutivos'' \cite[``Morfologia'' definicion 4]{rae2023Online}.

La morfología en el lenguaje ofensivo se refiere al estudio de cómo se forman y se utilizan las palabras ofensivas y las estructuras morfológicas que pueden contribuir a su carácter ofensivo.

\begin{itemize}

	\item Afijos ofensivos: Los afijos son morfemas que se añaden a una raíz o base para formar palabras. En el lenguaje ofensivo, pueden utilizarse afijos para crear términos peyorativos o insultantes. Por ejemplo, la adición de sufijos como ``-acho'', ``-ote'', o ``-ucho'' pueden cambiar el significado de una palabra de manera negativa o despectiva.
	
	\item Compuestos ofensivos: La morfología en el lenguaje ofensivo también puede involucrar la creación de palabras compuestas que combinan dos o más términos para formar expresiones ofensivas. Estos compuestos pueden ser especialmente creativos y directos en su ofensa.
	
	\item Reduplicación: En algunos casos, se utiliza la reduplicación de morfemas o sílabas para intensificar la ofensa. Por ejemplo, en inglés, la repetición de una sílaba en una palabra puede darle un tono más negativo, como ``idiota'' frente a ``idiooooota''.
	
	\item Derivación: La morfología de derivación se refiere a cómo se crean nuevas palabras a partir de palabras existentes mediante la adición de morfemas prefijos o sufijos. En el lenguaje ofensivo, la derivación puede utilizarse para crear términos insultantes a partir de palabras neutras. Por ejemplo, el prefijo ``des-'' o ``in-'' puede añadirse para crear palabras como ``desgraciado'' o ``incompetente''.
	
	\item Uso de diminutivos y aumentativos: En algunas ocasiones, el uso de diminutivos o aumentativos puede tener un efecto irónico o sarcástico en el lenguaje ofensivo. Por ejemplo, en español, el uso de ``itito'' o ``ote'' como diminutivos puede añadir un tono despectivo a una palabra, como ``gordito'' para referirse a alguien de manera ofensiva por su peso.
\end{itemize}