%%PREAMBULO SE RECOMIENDA NO MODIFICAR SOLO AGREGAR PAQUETES QUE NECESITEN\
\usepackage[justified]{ragged2e}
\usepackage{microtype}
\usepackage{caption}
\usepackage{amsmath}%
\usepackage{amsfonts}%
\usepackage{amssymb}%
\usepackage{graphicx}
%%Paquetes que siempre se usan
\usepackage[utf8]{inputenc}  
\usepackage[spanish]{babel}
\usepackage{csquotes}
\usepackage{color}
\usepackage[table]{xcolor}
% Paquete para la definición de nuevas colores
%\usepackage{xcolor}

\definecolor{cyan-keyword}{rgb}{0.0, 0.6, 0.8}
\definecolor{cyan-string}{rgb}{0.2, 0.6, 0.2}
\definecolor{cyan-comment}{rgb}{0.4, 0.4, 0.4}
\definecolor{cyan-function}{rgb}{0.0, 0.2, 0.4}
\definecolor{codegray}{rgb}{0.5,0.5,0.5}
\definecolor{miColorMuyClaro}{rgb}{0.9716, 0.9941, 0.9886}
\definecolor{orange}{rgb}{1.0, 0.95, 0.9}

\usepackage{listings}
\lstset{
	literate={á}{{\'a}}1
	{é}{{\'e}}1
	{í}{{\'i}}1
	{ó}{{\'o}}1
	{ú}{{\'u}}1
	{ñ}{{\~n}}1
	{Ñ}{{\~N}}1
	{·}{{$\cdot$}}1
	{ }{{$\cdot$}}1
}

\lstdefinestyle{mystyle}{
	backgroundcolor=\color{miColorMuyClaro},   
	commentstyle=\color{cyan-comment},
	keywordstyle=\bfseries\color{cyan-keyword},
	identifierstyle=\color{cyan-function},
	numberstyle=\tiny\color{codegray},
	stringstyle=\color{cyan-string},
	basicstyle=\ttfamily\small,
	lineskip=2pt, 
	columns=flexible,     
	breakatwhitespace=false,         
	breaklines=true,                 
	captionpos=b,                    
	keepspaces=true,                 
	numbers=left,                    
	numbersep=5pt,                  
	showspaces=false,                
	showstringspaces=false,
	showtabs=false,                  
	tabsize=2
}
\renewcommand{\lstlistlistingname}{Índice de Códigos Fuentes}
\renewcommand{\lstlistingname}{Código Fuente}
\lstset{style=mystyle}

\usepackage{graphicx}
\usepackage{epsfig}
\usepackage{epigraph}
\setlength\epigraphwidth{.8\textwidth}
\usepackage{lettrine}
\definecolor{lightblue}{rgb}{0.03,0.22,0.93}
\usepackage{tabularray}
%%notas para figura
\usepackage[capposition=bottom]{floatrow}
\usepackage{caption}

% Configuración para centrar la leyenda debajo de la imagen
\floatsetup[figure]{capposition=bottom, justification=centering}




\usepackage[absolute]{textpos} 
\setlength{\TPHorizModule}{\paperwidth}
\setlength{\TPVertModule}{\paperheight}
\newcommand{\tb}[4]{\begin{textblock}{#1}[0.5,0.5](#2,#3)\begin{center}#4\end{center}\end{textblock}}
%%%fijandoMargenes
\usepackage{geometry}
\geometry
{paperheight=279mm,%
paperwidth=215mm,%
top=20mm,%
left=30mm,%
right=20mm,%
bottom=20mm,%
headsep=7.5mm,%
footskip=10mm,%
textheight=217mm,%
textwidth=150mm,%
bindingoffset=10mm,}
%%%%%
\usepackage{mathptmx}
\usepackage{newtxtext,newtxmath}
\usepackage[shortlabels]{enumitem}
\usepackage{array}
\usepackage{multirow}
\usepackage{pifont}
\usepackage{pdfpages}
\usepackage{lipsum}
\usepackage{subcaption}
\usepackage{hyperref}
\usepackage{rotating}
\usepackage[export]{adjustbox}  %para justificar figuras
\usepackage{listings}  % PERMITE AGREGAR CÓDIGO DE LENGUAJES  DE PROGRAMACIÓN (DOCUMENTACIÓN EN GOOGLE)
\usepackage{emptypage}  % QUITA LOS ENCABEZADOS Y PIES DE PÁGINA EN LAS HOJAS VACÍAS PRODUCIDAS POR LA IMPRESIÓN A DOS CARAS
\usepackage{wrapfig}  % to include figure with text wrapping around it
\usepackage[margin=0pt,font=small,labelfont=bf]{caption}  % for improved layout of figure captions with extra margin, smaller font than text
%\usepackage[bf,SL,BF]{subfigure}  % Permite crear figuras múltiples
\usepackage{makeidx}  % Contiene los macros para indexar en un sario
%\usepackage[style=list,toc,number=none]{glossary}
\usepackage[refpages]{gloss} % para generar glosario a partir de una extension .bib
\usepackage{booktabs}      % para hacer tablas con: \midrule \toprule \bottomrule
\usepackage{multirow}      % para hacer tablas con: \multirow
\usepackage{mathdots}      % para el comando \iddots
\usepackage{mathrsfs}      % para formato de letra en ecuaciones
\raggedbottom              %Evita que LaTeX distribuya los espacios en blanco sobre la página, en lugar de eso los envía al fondo
\usepackage{fancyhdr}      % for better header layout
\usepackage{eucal}
\usepackage{float}
\usepackage{longtable} 
\usepackage{color, colortbl}

\usepackage[perpage]{footmisc}
\usepackage{ifthen}
\usepackage{multicol}       % for pages with multiple text columns, e.g. References 
\setlength{\columnsep}{10pt} % space between columns; default 10pt quite narrow
\usepackage[nottoc]{tocbibind} % correct page numbers for bib in TOC, nottoc suppresses an entry for TOC itself
\usepackage{nextpage}
\usepackage[compact]{titlesec}  
%\usepackage{titlesec}
\usepackage{lscape}     % for the orientation of the content in horizontal
\usepackage{pdflscape} % for the orientation of the pages in horizontal
\usepackage{pdfpages}  % para insertar archivos pdf
%\usepackage[siunitx]{circuitikz}  %para circuitos
%\usepackage[makeroom]{cancel} %Para cancelar términos en modo matemático
%\usepackage{cleveref}   %COMO UNA FORMA DE REFERENCIAR TABLAS, ECUACIONES, ETC. -->http://mirror.utexas.edu/ctan/macros/latex/contrib/cleveref/cleveref.pdf
\usepackage{rotating}        %para rotar el texto
\usepackage{setspace}
\usepackage{csquotes}
\usepackage{lipsum}
%%FIGURAS TABLAS APA
\DeclareCaptionLabelSeparator*{spaced}{\\[1.6ex]}
\captionsetup[table]{textfont=it,format=plain,justification=justified,
	singlelinecheck=false,labelsep=spaced,skip=04pt}
%\captionsetup[figure]{labelsep=period,labelfont=it,justification=justified,
%	singlelinecheck=false,font=doublespacing}
\captionsetup[figure]{textfont=it,format=plain,justification=justified,
	singlelinecheck=false,labelsep=spaced,skip=04pt}



%:Esquema de numeración por defecto
\setenumerate[1]{label=\normalfont\bfseries{\arabic*.}, leftmargin=*, labelindent=\parindent}
\setenumerate[2]{label=\normalfont\bfseries{\alph*}), leftmargin=*}
\setenumerate[3]{label=\normalfont\bfseries{\roman*.}, leftmargin=*}
\setlist{itemsep=0.1em} %Separacion de item en listas
\setlength{\parindent}{1.0 em}

\setcounter{tocdepth}{4}						% El nivel hasta el que se muestra el índice 
\renewcommand{\baselinestretch}{1.5}   %agregado el dia 12/06/2016
\usepackage{comment}

%Bibliografia
%\usepackage[language=spanish, backend=biber,style=apa]{biblatex}
\usepackage{apacite}
\usepackage{url}
%Agregando la fuente bibliografica
%\addbibresource{bibliografia/biblio.bib}  %AQUi agregamos nuestra base de datos

%%jUSTIFICACION A A LA IZQUIERDA
\raggedright

%%Numeracion ecuaciones: 

\counterwithout{equation}{chapter}
\renewcommand{\theequation}{Eq. \arabic{equation}}
