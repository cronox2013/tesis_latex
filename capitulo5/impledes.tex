La primera etapa consistió en la recopilación de datos relevantes para el proyecto. Para ello, se extrajeron aproximadamente 78,000 comentarios de diversas redes sociales populares en Bolivia, conformando un conjunto de datos representativo. Adicionalmente, se utilizó un conjunto de datos en español para apoyar ciertas tareas necesarias en el proyecto.

A continuación, se presenta el código fuente de cada archivo mencionado en la sección correspondiente del módulo:

El código fuente \ref{lst:modelo} corresponde al archivo manipular\_modelo.py, donde se puede apreciar la implementación del cargado del modelo, su respectiva capa embedding con keras y el tokenizador con el módulo pickle de python para el preprocesado del texto de entrada.

\lstinputlisting[language=Python,firstline=7,caption=Codigo fuente del archivo manipular\_modelo.py,label={lst:modelo}]{capitulo5/codigo/manipular_modelo.py}


El código fuente del archivo interfaz.py se desarrolló utilizando la librería Tkinter para crear una interfaz gráfica de usuario. Dado que es un componente secundario, no se proporcionarán detalles adicionales de este archivo.