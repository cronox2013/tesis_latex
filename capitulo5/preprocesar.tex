A continuación, se presenta el código fuente de cada archivo de las secciones detalladas anteriormente en el módulo de preprocesamiento de comentarios:

Limpieza de Datos

En la sección de limpieza de datos se almacenan dos archivos. El código fuente \ref{lst:c1} corresponde al archivo limpiar\_datos.py, donde se puede observar la implementación de la limpieza de datos extraídos de las redes sociales, se usaron expresiones regulares en la implementación de cada función. 


\lstinputlisting[language=Python,firstline=7,caption=Codigo fuente limpieza de datos,label={lst:c1}]{capitulo5/codigo/limpiar_datos.py}


El código fuente \ref{lst:c2} corresponde al archivo corrector\_lenguaje.py, que se encarga de la limpieza general de un conjunto de datos. Esto incluye la eliminación de URLs, caracteres especiales, números, entre otros elementos, utilizando la biblioteca re de Python.

\lstinputlisting[language=Python,firstline=7,caption=Codigo fuente limpieza de URLs,label={lst:c2}]{capitulo5/codigo/corrector_lenguaje.py}

Formato y Encadenamiento de Información

La segunda y última sección de este módulo también almacena dos archivos y se enfoca en el tratamiento de archivos y formatos. En el código fuente \ref{lst:c3} se puede observar la implementación del archivo manejo\_archivos.py, que se encarga de crear, modificar, eliminar, y gestionar archivos en distintos formatos.

\lstinputlisting[language=Python,firstline=7,caption=Codigo fuente limpieza archivos y formatos,label={lst:c3}]{capitulo5/codigo/manejo_archivos.py}

En el código fuente \ref{lst:c4} se puede observar la implementación del último archivo de esta sección, convertir\_formato.py. Este archivo se encarga de recorrer múltiples archivos para compactar la información y realizar su conversión posterior. Además, se lleva a cabo el registro de cada archivo utilizado y de cada proceso realizado con ellos.

\lstinputlisting[language=Python,firstline=8,caption=Codigo fuente compactacion de informacion,label={lst:c4}]{capitulo5/codigo/convertir_formato.py}

