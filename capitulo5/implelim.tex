A continuación, se presenta el código fuente de los archivos detallados anteriormente en esta sección:

%\textbf{Limpieza de Datos}

%El archivo de la carpeta limpiezadataset se detalla a continuación:

El código fuente \ref{lst:c1} corresponde al archivo limpiar\_datos.py, donde se puede observar la implementación de la limpieza de datos extraídos de las redes sociales, se usaron expresiones regulares en la implementación de cada función. 

\lstinputlisting[language=Python,firstline=7,caption=Codigo fuente del archivo limpiar\_datos.py,label={lst:c1}]{capitulo5/codigo/limpiar_datos.py}

\vspace{-1.3em} % Ajusta el valor según sea necesario

\begin{figure}[h!]
	\centering % Incrementa el contador de lstlisting
	\textit{Fuente: Elaboración propia}
\end{figure}






La carpeta gestionarchivos almacena dos archivos y se enfoca en el tratamiento de archivos y formatos. En el código fuente \ref{lst:c3} se puede observar la implementación del archivo manejo\_archivos.py, que se encarga de crear, modificar, eliminar, y gestionar archivos en distintos formatos.

\lstinputlisting[language=Python,firstline=7,caption=Codigo fuente del archivo manejo\_archivos.py,label={lst:c3}]{capitulo5/codigo/manejo_archivos.py}
\vspace{-1.3em} % Ajusta el valor según sea necesario

\begin{figure}[h!]
	\centering % Incrementa el contador de lstlisting
	\textit{Fuente: Elaboración propia}
\end{figure}

En el código fuente \ref{lst:c4} se puede observar la implementación del último archivo de esta sección, convertir\_formato.py. Este archivo se encarga de recorrer múltiples archivos para compactar la información y realizar su conversión posterior. Además, se lleva a cabo el registro de cada archivo utilizado y de cada proceso realizado con este.

\lstinputlisting[language=Python,firstline=8,caption=Codigo fuente del archivo convertir\_formato.py,label={lst:c4}]{capitulo5/codigo/convertir_formato.py}
\vspace{-1.3em} % Ajusta el valor según sea necesario

\begin{figure}[h!]
	\centering % Incrementa el contador de lstlisting
	\textit{Fuente: Elaboración propia}
\end{figure}