Las tareas de limpieza, preprocesamiento de datos, etiquetado, manejo de archivos y formatos, y la creación e implementación de los modelos de clasificación se realizaron utilizando los siguientes módulos, los cuales se detallan a continuación:

\begin{itemize}
	\item	Preprocesamiento de comentarios: Este módulo fue creado para llevar a cabo cualquier tarea de preprocesamiento de texto necesaria para los datos adquiridos, así como para el manejo de formatos y archivos.
	
	\item Etiquetado de comentarios: Este módulo se desarrolló para realizar el etiquetado de todos los comentarios recolectados y preprocesados, preparándolos para su uso en el modelo.

	\item Creación y entrenamiento de modelos: En este módulo se diseñaron y ajustaron los diferentes modelos clasificadores de lenguaje ofensivo basados en redes neuronales convolucionales propuestos.
	
	\item Clasificador de lenguaje ofensivo en el contexto boliviano: Para utilizar el modelo clasificador, se creó una aplicación de escritorio que carga el modelo generado y permite la entrada de texto para clasificarlo en una de las tres clases propuestas.

\end{itemize}

Para visualizar de manera más detallada la infraestructura que permite la limpieza, preprocesamiento, construcción del conjunto de datos, creación y uso del modelo clasificador de comentarios, se creó el diagrama de clases 5.1. En este diagrama se representa cada módulo y su interacción.	

-------------------------------------------
Diagrama de clases 5.1

-----------------------------------------