A continuación se detalla la implementación de cada archivo detallado en esta seccion:

En el código fuente \ref{lst:c9} se observa la implementación de todos los modelos de redes neuronales convolucionales basados en el modelo de tres capas convolucionales propuesto, contenidos en el archivo cnn\_tres.py.

\lstinputlisting[language=Python,firstline=7,caption=Codigo fuente del archivo cnn\_tres.py ,label={lst:c9}]{capitulo5/codigo/cnn_tres.py}
\vspace{-1.3em} % Ajusta el valor según sea necesario

\begin{figure}[h!]
	\centering % Incrementa el contador de lstlisting
	\textit{Fuente: Elaboración propia}
\end{figure}


En el código fuente \ref{lst:c10} se observa la implementación de los modelos propuestos con una capa convolucional mayor a la arquitectura base. Todo esto se encuentra en el archivo cnn\_cuatro.py

\lstinputlisting[language=Python,firstline=7,caption=Codigo fuente del archivo cnn\_cuatro.py ,label={lst:c10}]{capitulo5/codigo/cnn_cuatro.py}
\vspace{-1.3em} % Ajusta el valor según sea necesario

\begin{figure}[h!]
	\centering % Incrementa el contador de lstlisting
	\textit{Fuente: Elaboración propia}
\end{figure}


Finalmente, en el código fuente \ref{lst:c11} se observa la implementación de todos los modelos de redes neuronales convolucionales con dos capas, contenidos en el archivo cnn\_dos.py.

\lstinputlisting[language=Python,firstline=7,caption=Codigo fuente del archivo cnn\_dos.py ,label={lst:c11}]{capitulo5/codigo/cnn_dos.py}
\vspace{-1.3em} % Ajusta el valor según sea necesario

\begin{figure}[h!]
	\centering % Incrementa el contador de lstlisting
	\textit{Fuente: Elaboración propia}
\end{figure}



