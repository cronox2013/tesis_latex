A continuación, se presenta el código fuente de los  archivos detallados anteriormente:

En el código fuente \ref{lst:c7} se puede observar la implementación del primer archivo, creando\_dataset.py, que visualiza el estado del conjunto de datos y lo divide manualmente, priorizando una distribución equilibrada entre las clases.

\lstinputlisting[language=Python,firstline=8,caption=Codigo fuente del archivo creando\_dataset.py
\\\textit{Fuente: Elaboración Propia},label={lst:c7}]{capitulo5/codigo/creando_dataset.py}

En el código fuente \ref{lst:c8} se observa la implementación del archivo preparar\_datos.py, que prepara la entrada final lista para la primera capa de los modelos propuestos. Esto implica la representación del texto en formas numéricas, en este caso, mediante embeddings. Además contiene funciones de armado, compilación, entrenamiento, entre otras funciones para los modelos de redes convolucionales

\lstinputlisting[language=Python,firstline=9,caption=Codigo fuente del archivo preparar\_datos
\\\textit{Fuente: Elaboración Propia},label={lst:c8}]{capitulo5/codigo/preparar_datos.py}
