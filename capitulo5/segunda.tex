En esta iteración se detallan los resultados obtenidos por los modelos con cuatro capas convolucionales, los detalles de la arquitectura base utilizada para cada modelo se especificaron anteriormente en la sección de modelado. 
Los resultados obtenidos en el modelo base, los modelos con una técnica de regularización y los modelos con dos técnicas de regularización se detallan en la tabla \ref{tbl:10}.

\begin{table}[!ht]
	\centering
	\begin{tabular}{|c|c|c|c|c|c|}
		\hline
		\textbf{Nombre del modelo} & \textbf{Precisión} & \textbf{Perdida} & \textbf{Val\_Precisión} & \textbf{Val\_Perdida} & \textbf{Epoca} \\ \hline
		~ & 0.7580 & 0.57 & 0.7714 & 0.5561 & 4 \\ \cline{2-6}
		cnn\_four & 0.8678 & 0.33 & 0.7206 & 0.8668 & 13 \\ \cline{2-6}
		~ & 0.9729 & 0.05 & 0.6937 & 4.4606 & 150 \\ \hline
		~ & 0.7870 & 0.53 & 0.7611 & 0.5810 & 11 \\ \cline{2-6}
		cnn\_dp\_four & 0.8528 & 0.37 & 0.7501 & 0.6475 & 41 \\ \cline{2-6}
		~ & 0.9019 & 0.25 & 0.7297 & 0.8340 & 150 \\ \hline
		~ & 0.8158 & 0.46 & 0.7650 & 0.5887 & 31 \\ \cline{2-6}
		cnn\_dp\_four\_f & 0.8665 & 0.35 & 0.7430 & 0.6575 & 108 \\ \cline{2-6}
		~ & 0.8726 & 0.33 & 0.7263 & 0.7339 & 150 \\ \hline
		~ & 0.8013 & 0.80 & 0.7699 & 0.5947 & 45 \\ \cline{2-6}
		cnn\_dp\_four\_fi & 0.8203 & 0.45 & 0.7625 & 0.6179 & 79 \\ \cline{2-6}
		~ & 0.8300 & 0.41 & 0.73 & 0.66 & 150 \\ \hline
		~ & 0.7611 & 0.59 & 0.7442 & 0.6174 & 19 \\ \cline{2-6}
		cnn\_bndp\_four\_f & 0.8096 & 0.48 & 0.7314 & 0.6467 & 42 \\ \cline{2-6}
		~ & 0.8662 & 0.34 & 0.5514 & 2.0138 & 150 \\ \hline
		~ & 0.7890 & 0.53 & 0.7650 & 0.5887 & 55 \\ \cline{2-6}
		cnn\_bndp\_four\_fi & 0.8021 & 0.49 & 0.7630 & 0.6190 & 74 \\ \cline{2-6}
		~ & 0.8345 & 0.41 & 0.7403 & 0.6992 & 150 \\ \hline
		~ & 0.7768 & 0.62 & 0.7255 & 0.5657 & 128 \\ \cline{2-6}
		cnn\_bndp\_four\_ss & 0.7849 & 0.54 & 0.7349 & 0.6329 & 141 \\ \cline{2-6}
		~ & 0.7832 & 0.54 & 0.7282 & 0.6471 & 150 \\ \hline
	\end{tabular}
	\caption{Detalle resultados obtenidos}
	\label{tbl:10}
\end{table}

Los resultados no mostraron una mejora significativa en la precisión ni en la pérdida, por lo que se consideró innecesario desarrollar modelos más profundos. En su lugar, se decidió trabajar con un modelo menos profundo.
