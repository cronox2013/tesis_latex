El conjunto de datos utilizado en este proyecto fue recopilado exclusivamente de dos plataformas de redes sociales: Facebook y WhatsApp. Se extrajo una cantidad significativa de muestras, las cuales fueron sometidas a un exhaustivo proceso de preprocesamiento y análisis. Durante este proceso, se realizaron diversas operaciones de limpieza para garantizar la calidad y relevancia de los datos.

Como resultado de estas operaciones de limpieza, las cifras originales de muestras variaron. A continuación, se presentan dos tablas que resumen este proceso: La tabla \ref{tbl:14} muestra la cantidad inicial de comentarios extraídos, incluyendo aquellos que contenían enlaces, duplicados y contenido no relevante, entre otros aspectos. La tabla \ref{tbl:15} presenta la cantidad final de comentarios considerados útiles después de haber sido sometidos al proceso de limpieza. Se eliminaron duplicados y se aplicaron filtros para garantizar la calidad y relevancia de los datos.

\begin{table}[!ht]
	\centering
	\begin{tabular}{|c|c|}
		\hline
		\textbf{Fuente} & \textbf{Numero de Comentarios} \\ \hline
		Santa Cruz & 16380 \\ 
		La Paz & 15861 \\ 
		Cbba & 38913 \\ 
		WhatsApp & 20826 \\ \hline
		\textbf{Total comentarios} & 91980 \\ \hline
	\end{tabular}
	\caption{Detalle cifras originales de comentarios extraídos
		\\\textit{Fuente: Elaboracion Propia}}
	\label{tbl:14}
\end{table}

\begin{table}[!ht]
	\centering
	\begin{tabular}{|c|c|}
		\hline
		\textbf{Fuente} & \textbf{Numero de Comentarios} \\ \hline
		Santa Cruz & 14385 \\ 
		La Paz & 14229 \\ 
		Cbba & 35863 \\ 
		WhatsApp & 14501 \\ \hline
		\textbf{Total comentarios} & 78978 \\ \hline
	\end{tabular}
	\caption{Detalle cifras de comentarios preprocesados extraídos
		\\\textit{Fuente: Elaboracion Propia}}
	\label{tbl:15}
\end{table}

Los comentarios cuyas cifras han sido presentadas en las tablas \ref{tbl:14} y \ref{tbl:15} fueron cuidadosamente seleccionados de las redes sociales, para reflejar la diversidad y riqueza de la comunicación en cada uno de los nueve departamentos de Bolivia. Estos departamentos están divididos en tres regiones distintas: el altiplano, que incluye La Paz, Oruro y Potosí; los valles, que abarcan Chuquisaca, Cochabamba y Tarija; y los llanos, que comprenden Santa Cruz, Pando y Beni.

Esta selección se llevó a cabo considerando variantes existentes en el uso del idioma español en cada región del país. Además, se decidió focalizar en un departamento específico de cada zona debido a la distribución demográfica característica de cada uno. Por ejemplo, los departamentos de Santa Cruz, La Paz y Cochabamba albergan la mayor cantidad de habitantes en sus respectivas regiones, lo que los convierte en representantes significativos de la diversidad lingüística y cultural de Bolivia. A continuación se brindarán detalles sobre la recolección de comentarios en las redes sociales de facebook y whatsapp en estas regiones:


%\begin{itemize}
%\item Palabras clave y elección de comentarios
%\end{itemize}
%A continuación se brindarán detalles sobre la recolección de comentarios en las redes sociales de facebook y whatsapp.

\begin{itemize}

\item Facebook

Todos los comentarios de la red social Facebook fueron extraídos de forma manual, empleando palabras o frases clave mas los nombres de los tres departamentos de Bolivia seleccionados. Esta práctica se llevó a cabo debido a la falta de un control sobre la ubicación geográfica de las publicaciones en Facebook. A continuación se presentan las palabras clave seleccionadas:
\begin{itemize}
	\item racista /racismo
	\item sexista/sexismo
	\item homosexual/homofobia/homofobico
	\item lenguaje ofensivo/lenguaje de odio 
	\item discriminar/discriminacion
	\item machista/machismo
	\item violento/violencia
	\item feminista/feminismo  
\end{itemize}

Además de utilizar palabras clave para identificar publicaciones relevantes para este proyecto, también se llevó a cabo la selección de perfiles de autoridades políticas, medios de información y medios de comunicación en Bolivia. Esto se hizo con la comprensión de que los temas políticos y los hechos relevantes del país siempre han sido de gran interés para la población boliviana. Es en estos perfiles donde las personas tienden a concentrar sus opiniones y expresar sus desacuerdos de manera más frecuente. Para más detalles ver tabla \ref{tbl:16}.


\begin{table}[!ht]
	\centering
	\begin{tabular}{|c|c|}
		\hline
		\textbf{Tipo de perfil} & \textbf{Nombre de Perfiles} \\ \hline
		Figuras politicas del pais  & \makecell{Evo Morales Ayma, Luis Fernando Camacho, \\  Andronico Rodriguez} \\ \hline
		Periódicos digitales                       & El Deber, Los tiempos, Pagina siete \\ \hline
		Canales de television & Unitel, Atb, Bolivision \\ \hline
		Radio  & Radio Qhana, Radio Sonar \\ \hline
		Otros medios de comunicacion & Sport Bolivia, Mi bolivia Plurinacional \\ \hline
	\end{tabular}
	\caption{Detalle perfiles para extraccion de comentarios
		\\\textit{Fuente: Elaboracion Propia}}
	\label{tbl:16}
\end{table}

\item{Whatsapp}


Para recolectar los comentarios de la aplicación de mensajería WhatsApp, se eligió un grupo de chat compuesto por 7 miembros jóvenes, con edades comprendidas entre los 21 y 27 años, que habitualmente se comunicasen de manera brusca, grosera y/o ofensiva. Esta selección se realizó con el consentimiento del administrador del grupo, quien exportó el chat directamente desde WhatsApp en un documento con extensión .txt. Posteriormente, el archivo fue sometido a un proceso exhaustivo de limpieza y preprocesamiento de datos, los detalles de la limpieza y la clasificación de los mismos se detallaran más  adelante.

\end{itemize}




