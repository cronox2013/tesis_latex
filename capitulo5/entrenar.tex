Este módulo se divide en una sección con cinco archivos, encargada de todo el proceso necesario para entrenar los modelos de redes convolucionales.

Entrenamiento de modelos

A continuación se detalla cada archivo de la sección entrenamiento de modelos:

En el código fuente \ref{lst:c7} se puede observar la implementación del primer archivo, creando\_dataset.py, que visualiza el estado del conjunto de datos y lo divide manualmente, priorizando una distribución equilibrada entre las clases.

\lstinputlisting[language=Python,firstline=8,caption=Codigo fuente visualizacion estado del conjunto de datos,label={lst:c7}]{capitulo5/codigo/creando_dataset.py}

En el código fuente \ref{lst:c8} se observa la implementación del archivo preparar\_datos.py, que prepara la entrada final lista para la primera capa de los modelos propuestos. Esto implica la representación del texto en formas numéricas, en este caso, mediante embeddings. Además contiene funciones de armado, compilación, entrenamiento, entre otras funciones para los modelos de redes convolucionales

\lstinputlisting[language=Python,firstline=9,caption=Codigo fuente preparacion entrada final,label={lst:c8}]{capitulo5/codigo/preparar_datos.py}

En el código fuente \ref{lst:c9} se observa la implementación de todos los modelos de redes neuronales convolucionales basados en el modelo base propuesto, contenido en el archivo cnn\_tres.py.

\lstinputlisting[language=Python,firstline=7,caption=Codigo fuente  implementación de modelos de redes neuronales convolucionales,label={lst:c9}]{capitulo5/codigo/cnn_tres.py}

En el código fuente \ref{lst:c10} se observa la implementación de los modelos propuestos con diferentes arquitecturas a la arquitectura base para mejorar la precisión en la predicción de los otros modelos. Todo esto se encuentra en el archivo cnn\_cuatro.py.

\lstinputlisting[language=Python,firstline=7,caption=Codigo fuente implementación de los modelos con diferentes arquitecturas,label={lst:c10}]{capitulo5/codigo/cnn_cuatro.py}

Finalmente, en el código fuente \ref{lst:c11} se observa la implementación de todos los modelos de redes neuronales convolucionales con dos capas, contenido en el archivo cnn\_dos.py.

\lstinputlisting[language=Python,firstline=7,caption=Codigo fuente implementación de redes neuronales convolucionales con dos capas,label={lst:c11}]{capitulo5/codigo/cnn_dos.py}

