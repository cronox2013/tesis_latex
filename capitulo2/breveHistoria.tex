Las raíces de las redes neuronales convolucionales (CNN, por sus siglas en inglés Convolutional Neural Networks) se remontan a la década de 1950, cuando entre 1958 y 1959 David Hubel y Torsten Wiesel investigaron la corteza visual realizando experimentos en  gatos. Sus estudios revelaron que muchas de las neuronas en esta región cerebral, tienen pequeños  campos  receptivos locales que reaccionan solamente a estímulos visuales,  ubicados en una región limitada del campo visual.\cite{hubel1959single} Además algunas de estas neuronas  reaccionaban solamente a imágenes con líneas verticales y algunas otras a líneas con distintas orientaciones sin importar que, tuviesen el mismo campo receptivo.  Las neuronas que poseian un campo receptivo más grande, podían reaccionar a patrones más complejos, que eran la combinación de  patrones más simples de un nivel inferior. Este tipo de comportamiento,  hizo que se concluyera a la idea de que las neuronas de nivel superior, se basan en las salidas de las neuronas vecinas de nivel inferior. \cite{hubel1959receptive}

Este trabajo meritorio del Premio Nobel de Fisiología o Medicina inspiró a Fukushima, quien en la década de 1980 propuso la primera arquitectura de red neuronal convolucional conocida como ``Neocognitron''. Este modelo fue capaz de reconocer patrones visuales y estaba compuesto por cuatro tipos de capas entre las cuales resaltaban dos: las capas S (submuestreo) reducían la dimensionalidad de la entrada al tomar secciones de una imagen y reducir su tamaño, preservando las características más relevantes. Las capas C (convolución) eran  responsables de aplicar operaciones de convolución a las entradas. En estas capas, cada neurona estaba conectada únicamente a un área local de las capas previas, reflejando la organización de los campos receptivos locales en las células visuales  \cite{fukushima1980neocognitron}. 

A pesar de estos avances, las CNN no recibieron mayor atención y aplicación práctica hasta la década de 1990, especialmente en el ámbito del reconocimiento de patrones e imágenes. Es en esta época cuando comienzan a destacar y a encontrar aplicaciones prácticas significativas.

El punto de inflexión llegó en 1998, cuando Yann LeCun, junto con otros investigadores, desarrolló una CNN conocida como LeNet-5 para el reconocimiento de dígitos escritos a mano. LeNet-5 fue especialmente revolucionaria porque demostró una alta precisión en la clasificación de dígitos en documentos de cheques bancarios \cite{lecun1998gradient}.

A lo largo de los años, la capacidad de las CNN para procesar datos de imagen se vio fortalecida por la disponibilidad de grandes conjuntos de datos etiquetados, mejoras en hardware (como GPUs) que aceleraron los cálculos intensivos y avances en técnicas de entrenamiento como el uso de funciones de activación no lineales (ReLU) y regularización.

El avance más reciente y significativo de las CNN se produjo con la participación en competiciones de reconocimiento de imágenes, como ImageNet, donde en 2012, un equipo dirigido por Geoffrey Hinton utilizó una CNN denominada ``AlexNet'', que superó ampliamente a otros métodos existentes y estableció un nuevo estándar en el campo de la visión por computadora \cite{krizhevsky2012imagenet}. Desde entonces, las CNN han sido fundamentales en numerosas aplicaciones de reconocimiento de imágenes, segmentación, detección de objetos, entre otros, y continúan siendo objeto de investigación para mejorar su rendimiento y aplicabilidad en una variedad de tareas.
