Durante el proceso de extracción o adquisición de datos textuales, es común encontrarse con texto sin formato que contiene información no textual irrelevante, como marcas HTML, especialmente cuando se extrae de páginas web. Por lo tanto, la limpieza de texto se refiere a la transformación de este texto en un formato consumible para procesos posteriores.

En fuentes de contenido textual, puede encontrarse una variedad incontrolable de elementos como signos, marcas, metadatos, números, etc. Para modelos de aprendizaje automático, es crucial disponer de una cantidad abundante de texto para una mejor generalización. Sin embargo, al recolectar datos textuales, no siempre es posible seleccionar únicamente el texto deseado. Por lo tanto, después de la adquisición del texto, es necesario implementar un sistema que extraiga eficazmente el texto requerido y elimine cualquier contenido no deseado. Los pasos generales que se pueden seguir son los siguientes:

\begin{itemize}

\item Identificar el texto requerido.
\item Localizar las marcas de inicio y fin del texto deseado.
\item Identificar y eliminar tipos de texto innecesarios para reducir el volumen de datos.
\item Extraer el texto contenido entre las marcas de inicio y fin.
\item Guardar el texto en un formato adecuado para su uso posterior.

\end{itemize}

Después de realizar la limpieza inicial del texto, es crucial identificar cualquier contenido irrelevante y ruido restante dentro del texto. Esta tarea forma parte de la fase de preprocesamiento del texto.