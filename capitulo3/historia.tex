El Procesamiento del Lenguaje Natural (PLN) ha evolucionado desde los años 50, cuando Alan Turing planteó la posibilidad de evaluar la inteligencia de una máquina a través de conversaciones humanas en un artículo publicado en 1950 llamado ``Computing Machinery and Intelligence'' (Máquinas de computación e inteligencia) \cite{turing2009computing}. En la década de los 60, se desarrollaron los primeros programas de traducción automática y análisis gramatical. Uno de los primeros sistemas de PLN fue el ``ELIZA'', creado por Joseph Weizenbaum en 1966, que simulaba una conversación terapéutica \cite{weizenbaum1966eliza}.

En los años 70, la influencia de la Gramática Generativa Transformacional de Noam Chomsky fue significativa en el ámbito del Procesamiento del Lenguaje Natural (PLN). Esta teoría lingüística propuesta por Chomsky en las décadas anteriores buscaba establecer reglas abstractas y universales para explicar cómo se generan y estructuran las oraciones en cualquier lengua \cite{chomsky1970remarks}.

En ese período, los modelos de PLN estaban fuertemente influenciados por esta perspectiva basada en reglas. Se intentaba aplicar las ideas de la gramática generativa a la comprensión del lenguaje natural mediante la creación de sistemas computacionales que imitaran, en cierta medida, las reglas y principios gramaticales identificados por Chomsky. Estos sistemas se centraban en la construcción de reglas para analizar y procesar el lenguaje.

En los 80 hubo un cambio en el enfoque utilizado en los sistemas de PLN. Se adoptaron sistemas basados en el conocimiento y la codificación manual de reglas. En lugar de depender únicamente de los principios gramaticales generales, se codificaba manualmente reglas y patrones lingüísticos que se consideraban relevantes para las tareas específicas

El avance hacia los 90 trajo consigo la introducción de métodos estadísticos y basados en el aprendizaje automático en el PLN, junto con el crecimiento de la web que impulsó la necesidad de herramientas de búsqueda y análisis de texto a gran escala.

Desde el año 2000 en adelante, el PLN se ha expandido enormemente y se ha convertido en un campo multidisciplinario que incorpora el aprendizaje profundo, modelos basados en la estadística, el procesamiento de grandes volúmenes de datos (big data) y la aplicación de técnicas avanzadas en áreas como la comprensión del lenguaje, la traducción automática y los chatbots.

Esta rápida evolución continúa en la actualidad, con el PLN jugando un papel crucial en numerosas aplicaciones, desde motores de búsqueda hasta asistentes virtuales y análisis de sentimientos en redes sociales.
