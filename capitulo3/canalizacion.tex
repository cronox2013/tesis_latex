En el desarrollo de soluciones en PLN, es habitual desglosar el problema en componentes más pequeños y manejables. Esto implica identificar los requisitos esenciales y dividir el problema en varios subproblemas, cada uno de los cuales se aborda de manera individual.

Este enfoque incluye un detallado procesamiento de texto en cada fase, el mismo es conocido como canalización del procesamiento de lenguaje natural y representa la serie de pasos necesarios para construir cualquier modelo de PLN. Estos pasos se encuentran omnipresentes en los proyectos de PLN, convirtiéndolos en un requisito en este campo, la visualización de estas etapas clave del proceso, su relación y orden se puede observar en la figura 2.1.

-------------------------------------------

figura 2.1

-----------------------------------------

A continuación se detalla específicamente todo este proceso paso por paso, para el tratamiento de texto y su uso en modelos de aprendizaje automático.

