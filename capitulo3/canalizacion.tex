En el desarrollo de soluciones en el PLN, es habitual desglosar el problema en componentes más pequeños y manejables. Esto implica identificar los requisitos esenciales y dividir el problema en varios subproblemas, cada uno de los cuales se aborda de manera individual.

Este enfoque incluye un detallado procesamiento de texto en cada fase. Este procesamiento secuencial del texto, conocido como canalización, representa la serie de pasos necesarios para construir cualquier modelo de PLN. Estos pasos se encuentran omnipresentes en los proyectos de PLN, convirtiéndolos en un requisito en este campo, es por eso que a continuación se detalla específicamente todo el proceso requerido paso por paso, para el tratamiento de texto.
