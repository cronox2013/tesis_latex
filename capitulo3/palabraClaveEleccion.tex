A continuación se brindarán detalles sobre la recolección de comentarios en las redes sociales de facebook y whatsapp.

\textbf{Facebook}

Todos los comentarios de la red social Facebook fueron extraídos de forma manual, empleando palabras o frases clave durante la búsqueda de las publicaciones correspondientes. Esta práctica se llevó a cabo debido a la falta de un control sobre la ubicación geográfica de las publicaciones en Facebook. A continuación se presentan las palabras clave seleccionadas:
\begin{itemize}
	\item racista /racismo
	\item sexista/sexismo
	\item homosexual/homofobia/homofobico
	\item lenguaje ofensivo/lenguaje de odio 
	\item discriminar/discriminacion
	\item machista/machismo
	\item violento/violencia
	\item feminista/feminismo  
\end{itemize}
y los nombres de las ciudades de Bolivia que se usaron conjuntamente con cada uno de las palabras clave en la búsqueda para encontrar comentarios ofensivos en la red social facebook:
\begin{itemize}
	\item Cochabamba
	\item Santa Cruz
	\item La Paz
	
\end{itemize}

Además de utilizar palabras clave para identificar publicaciones relevantes para este proyecto, también se llevó a cabo la selección de perfiles de autoridades políticas, medios de información y medios de comunicación en Bolivia. Esto se hizo con la comprensión de que los temas políticos y los hechos relevantes del país siempre han sido de gran interés para la población boliviana. Es en estos perfiles donde las personas tienden a concentrar sus opiniones y expresar sus desacuerdos de manera más frecuente. Para mas detalles ver tabla \ref{tbl:16}.


\begin{table}[!ht]
	\centering
	\begin{tabular}{|c|c|}
		\hline
		\textbf{Tipo de perfil} & \textbf{Nombre de Perfiles} \\ \hline
		Figuras politicas del pais  & \makecell{Evo Morales Ayma, Luis Fernando Camacho, \\  Andronico Rodriguez} \\ \hline
		Periódicos digitales                       & El Deber, Los tiempos, Pagina siete \\ \hline
		Canales de television & Unitel, Atb, Bolivision \\ \hline
		Radio  & Radio Qhana, Radio Sonar \\ \hline
		Otros medios de comunicacion & Sport Bolivia, Mi bolivia Plurinacional \\ \hline
		~ & ~ \\ \hline
	\end{tabular}
	\caption{Detalle perfiles para extraccion de comentarios}
	\label{tbl:16}
\end{table}

\textbf{Whatsapp}


Para recolectar los comentarios de la aplicación de mensajería WhatsApp, se eligió un grupo de chat compuesto por 7 miembros jóvenes, con edades comprendidas entre los 21 y 27 años, que habitualmente se comunicasen de manera brusca, grosera y/o ofensiva. Esta selección se realizó con el consentimiento del administrador del grupo, quien exportó el chat directamente desde WhatsApp en un documento con extensión .txt. Posteriormente, el archivo fue sometido a un proceso exhaustivo de limpieza y preprocesamiento de datos, los detalles de la limpieza y la clasificación de los mismos se detallaran más  adelante.
