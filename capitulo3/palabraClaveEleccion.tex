A continuación se brindarán detalles sobre la recolección de comentarios en las redes sociales de facebook y whatsapp.

Facebook

Todos los comentarios de la red social Facebook fueron extraídos de forma manual, empleando palabras o frases clave durante la búsqueda de las publicaciones correspondientes. Esta práctica se llevó a cabo debido a la falta de un control sobre la ubicación geográfica de las publicaciones en Facebook. A continuación se presentan las palabras clave seleccionadas, con la probabilidad de encontrar comentarios ofensivos en la red social, como se detalla en la Tabla 1:

tala1
-----------------------------------



----------------------------------


Además de utilizar palabras clave para identificar publicaciones relevantes para este proyecto, también se llevó a cabo la selección de perfiles de autoridades políticas, medios de información y medios de comunicación en Bolivia. Esto se hizo con la comprensión de que los temas políticos y los hechos relevantes del país siempre han sido de gran interés para la población boliviana. Es en estos perfiles donde las personas tienden a concentrar sus opiniones y expresar sus desacuerdos de manera más frecuente. Para mas detalles ver tabla 2


tabla 2
------------------------------------




----------------------------------

Whatsapp


Para recolectar los comentarios de la aplicación de mensajería WhatsApp, se eligió un grupo de chat compuesto por 7 miembros jóvenes, con edades comprendidas entre los 21 y 27 años, que habitualmente se comunicasen de manera brusca, grosera y/o ofensiva. Esta selección se realizó con el consentimiento del administrador del grupo, quien exportó el chat directamente desde WhatsApp en un documento con extensión .txt. Posteriormente, el archivo fue sometido a un proceso exhaustivo de limpieza y preprocesamiento de datos, los detalles de la limpieza y la clasificación de los mismos se detallaran más  adelante.
