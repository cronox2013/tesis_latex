El Procesamiento del Lenguaje Natural (PLN) constituye un campo multidisciplinario que se centra en la interacción entre las computadoras y el lenguaje humano. Su objetivo principal es ayudar a las máquinas a comprender, interpretar y generar lenguaje natural de manera similar a la capacidad humana. En este capítulo, se proporciona una visión detallada que abarca desde los fundamentos teóricos hasta las técnicas modernas y las aplicaciones prácticas del PLN, lo que permite comprender tanto el estado actual como el potencial de este campo. Además, dada la escasez de conjuntos de datos que contienen lenguaje ofensivo en español, particularmente aquellos enfocados en el español de Bolivia, se detalla la creación de un conjunto de datos específico. Este conjunto de datos resultara sumamente útil para ser empleado por los modelos de aprendizaje automático propuestos en este contexto.
