El Procesamiento del Lenguaje Natural  abarca una gran variedad de tareas que se han vuelto imprescindibles para las personas, las mismas que  nos muestran una increíble evolución en la aplicabilidad de esta disciplina . Algunas de las tareas más comunes en PLN son:

\begin{itemize}

	\item Clasificación de Texto: Esta es una de las tareas más populares, se encarga de asignar etiquetas o categorías a un texto dado. Ejemplos incluyen la detección de spam en correos electrónicos o la clasificación de noticias por tema.

	\item Extracción de Información: Implica identificar y extraer información específica de textos, como nombres de personas, ubicaciones, fechas, etc., desde artículos o documentos.

	\item Análisis de Sentimientos: Determina la actitud emocional detrás de un texto, ya sea positiva, negativa o neutral. Es útil para evaluar opiniones en redes sociales, reseñas de productos, etc.

	\item Generación de Lenguaje Natural: Crear texto coherente y significativo, como resúmenes automáticos, diálogos generativos, entre otros.

	\item Reconocimiento de Entidades Nombradas (NER): identificar y clasificar entidades importantes en un texto, como nombres de personas, lugares, organizaciones, etc.

	\item Traducción automática: Convertir texto de un idioma a otro, lo que implica comprender y generar texto en diferentes lenguajes. Por ejemplo, la herramienta más popular que realiza la tarea de traducir actualmente más 100 idiomas es Google Translate.
	
	\item Modelado de Lenguaje: Esta tarea se encarga de  predecir la probabilidad de una secuencia de palabras en un idioma determinado, lo que se usa en corrección gramatical, autocompletado de texto, entre otros. 

\end{itemize}

Estas tareas son solo una muestra de las aplicaciones del PLN, cada una de ellas tiene su nivel de dificultad y su conjunto específico de métodos de resolución.

