El conjunto de datos utilizado en este proyecto fue recopilado exclusivamente de dos plataformas de redes sociales: Facebook y WhatsApp. Se extrajo una cantidad significativa de muestras, las cuales fueron sometidas a un exhaustivo proceso de preprocesamiento y análisis. Durante este proceso, se realizaron diversas operaciones de limpieza para garantizar la calidad y relevancia de los datos.

Como resultado de estas operaciones de limpieza, las cifras originales de muestras variaron. A continuación, se presentan dos tablas que resumen este proceso: La tabla 1 muestra la cantidad inicial de comentarios extraídos, incluyendo aquellos que contenían enlaces, duplicados y contenido no relevante, entre otros aspectos. La tabla presenta la cantidad final de comentarios considerados útiles después de haber sido sometidos al proceso de limpieza. Se eliminaron duplicados y se aplicaron filtros para garantizar la calidad y relevancia de los datos.

tabla 1
-------------------------------------------------------------
cifras originales de comentarios extraidos:
Conjunto de datos
Santa Cruz   16380
La Paz          15861
Cbba             38913
WhatsApp     20826
total comentarios 91980
cifras finales despues de la limpieza
---------------------------------------------------------------

tabla 2
-----------------------------------------------------------
Conjunto de datos
Santa Cruz  14385
La Paz 14229
Cbba 35863
WhatApp 14501
total comentarios 78978
-----------------------------------------------------------

Los comentarios cuyas cifras han sido presentadas en las tablas anteriores fueron cuidadosamente seleccionados para reflejar la diversidad y riqueza de la comunicación en cada uno de los nueve departamentos de Bolivia. Estos departamentos están divididos en tres regiones distintas: el altiplano, que incluye La Paz, Oruro y Potosí; los valles, que abarcan Chuquisaca, Cochabamba y Tarija; y los llanos, que comprenden Santa Cruz, Pando y Beni.

Esta selección se llevó a cabo considerando variantes existentes en el uso del idioma español en cada región del país. Además, se decidió focalizar en un departamento específico de cada zona debido a la distribución demográfica característica de cada uno. Por ejemplo, los departamentos de Santa Cruz, La Paz y Cochabamba albergan la mayor cantidad de habitantes en sus respectivas regiones, lo que los convierte en representantes significativos de la diversidad lingüística y cultural de Bolivia.


