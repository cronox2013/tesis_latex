Tanto en Bolivia como en otras partes del mundo, el uso generalizado de las redes sociales ha transformado la forma en que nos comunicamos y compartimos información. Sin embargo, este avance tecnológico también ha llevado a un aumento preocupante en el lenguaje ofensivo y el contenido perjudicial en línea, lo que afecta negativamente la experiencia de los usuarios y la calidad del discurso público. La relevancia de este proyecto radica en la necesidad de abordar este problema creciente y mejorar el ambiente en línea, promoviendo un espacio más seguro y respetuoso para la comunidad virtual boliviana.

La aplicación de redes neuronales convolucionales, es una técnica de aprendizaje profundo que, aplicada para la detección automatizada de lenguaje ofensivo representa una solución tecnológica eficaz. Siendo una técnica probada en otros contextos, su implementación específica en el entorno de redes sociales en Bolivia es un paso significativo hacia la mejora de la moderación de contenido en línea. Esta tecnología tiene el potencial de identificar y filtrar de manera eficiente el lenguaje ofensivo, promoviendo así una comunicación más constructiva y respetuosa en las plataformas de redes sociales utilizadas en el país.

Además de su contribución tecnológica, este proyecto también tiene un impacto social significativo al abordar cuestiones relacionadas con la seguridad en línea, la salud mental de los usuarios y la mejora del discurso público. También, contribuirá al cuerpo de conocimientos en el ámbito de la detección de lenguaje ofensivo en un contexto boliviano, lo que beneficiará a la sociedad en general. En última instancia, este proyecto se alinea con la necesidad imperante de construir un entorno digital más inclusivo, respetuoso y seguro en Bolivia.
