En la era digital, las redes sociales se han convertido en un espacio fundamental para la interacción, el debate y la expresión de opiniones en Bolivia, al igual que en todo el mundo. Sin embargo, uno de los problemas más apremiantes que enfrentan los usuarios de las redes sociales en Bolivia es la creciente presencia de comentarios con lenguaje ofensivo y contenido insultante en estas plataformas. Gran parte del problema es la ventaja que sacan estos usuarios del anonimato en las cuentas falsas que crean para expresarse de esta manera, este tipo de situaciones son alentadas al quedarse impunes, al no tener mucho o nada que hacer para tratar de corregirse y por ende no recibir ningún tipo de penalización.
Este problema plantea preocupaciones significativas desde diversas perspectivas:
\begin{itemize}
	\item	Impacto en la Sociedad.- La proliferación de comentarios ofensivos en las redes sociales tiene un impacto negativo en la calidad del discurso público en Bolivia. Contribuye a la polarización, la intolerancia, la discriminación, el racismo, la violencia y el conflicto en línea, socavando la posibilidad de un diálogo constructivo y respetuoso.
	\item Salud Mental y Bienestar.- Los comentarios ofensivos pueden tener un impacto perjudicial en la salud mental y el bienestar de los usuarios que son objeto de ataques verbales. Pueden experimentar estrés, ansiedad,  depresión, perpetuar el estigma en torno al suicidio y muchos otros efectos negativos graves.
	\item Derechos y Ética.- El lenguaje ofensivo y los comentarios insultantes pueden infringir los derechos individuales de los usuarios en línea y plantear cuestiones éticas relacionadas con la libertad de expresión y la responsabilidad en línea.
	\item Percepción Pública.- La presencia de comentarios ofensivos en las redes sociales también puede dañar la percepción pública de Bolivia, afectando a la reputación del país en el ámbito internacional.	
\end{itemize}
Este tipo de comentarios que solamente provocan efectos negativos en la sociedad deberían ser considerados una amenaza no tolerable que afectan a la integridad y el bienestar de una sociedad entera, deberían ser detectados y eliminados a tiempo para que no puedan causar más daño.