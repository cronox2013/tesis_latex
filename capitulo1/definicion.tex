Este proyecto aborda un problema central: la presencia y aumento alarmante de lenguaje ofensivo y 
contenido agresivo en las redes sociales en Bolivia. En el contexto boliviano, se ha identificado un preocupante arraigo de contenido ofensivo que se remonta a la época colonial, cuando un sistema de jerarquía basado en la pigmentación de la piel, castas y clases sociales estaba profundamente arraigado en el país. Además, se ha observado un fuerte rechazo hacia minorías con preferencias sexuales no convencionales, así como contenido altamente ofensivo relacionado con preferencias políticas, lo que ha llevado a la segregación en grupos y comunidades virtuales.

Aunque los tipos y enfoques de este contenido no difieren significativamente de lo que se encuentra en otras poblaciones, la característica distintiva radica en la utilización de la cultura y la jerga boliviana para crear este tipo de contenido. La intersección de estas cuestiones culturales y sociales plantea desafíos únicos que este proyecto busca abordar para fomentar un ambiente en línea más respetuoso y equitativo en Bolivia.
