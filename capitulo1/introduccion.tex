	   
Con el creciente número de personas con acceso a internet, los usuarios de las redes sociales se han incrementado por millones, las mismas, impulsadas por plataformas y aplicaciones en línea, han  permitido a las personas conectarse, comunicarse y compartir información de manera virtual. Si bien algunas personas realizan un uso adecuado de las redes sociales, existen también otras que le dan un uso totalmente reprochable, con sus acciones buscan causar daño, incomodidad y/o malestar en los demás. Estas personas utilizan el lenguaje ofensivo para dañar a otras personas, este se refiere a la utilización de palabras, frases o expresiones que son insultantes, hirientes, denigrantes o irrespetuosas, este tipo de lenguaje puede manifestarse de diversas formas en diferentes países, dependiendo del contexto cultural y lingüístico del país de estudio.

 El presente proyecto tiene como objetivo la realización de un conjunto de datos efectivo, donde se pueda tener ejemplos valiosos sobre el lenguaje ofensivo. Este conjunto de datos pretende representar los ejemplos necesarios que tomen en cuenta los aspectos necesarios para comprender la manera en la que se representa el lenguaje ofensivo en el país de Bolivia. Además se busca crear un modelo de red neuronal convolucional que se pueda adaptar a las características del lenguaje ofensivo, para realizar la mejor detección posible de estos comentarios que utilicen el mismo.