Este proyecto se centrará en desarrollar e implementar un sistema basado en redes neuronales convolucionales para la detección y clasificación automatizada del lenguaje ofensivo en las redes sociales utilizadas en Bolivia. El alcance del proyecto incluirá la recopilación de un conjunto de datos representativo de contenido ofensivo en línea específico de Bolivia, la construcción y entrenamiento de un modelo de CNN adaptado a este conjunto de datos, y la evaluación de su rendimiento en la clasificación precisa de lenguaje ofensivo en el contexto boliviano. Además, se considerarán aspectos éticos y de privacidad relacionados con la recolección y el uso de datos sensibles.

Este proyecto tiene ciertas limitaciones importantes. En primer lugar, la efectividad del modelo de CNN podría verse influenciada por la evolución rápida del lenguaje en línea, lo que puede requerir actualizaciones periódicas para mantener su precisión. Además, la detección de lenguaje ofensivo en contextos culturales y lingüísticos específicos, como Bolivia, presenta desafíos adicionales debido a la diversidad de dialectos y expresiones regionales. Además, la exactitud del modelo puede verse afectada por la calidad y representatividad del conjunto de datos disponible. También es importante destacar que, si bien el proyecto puede identificar lenguaje ofensivo, no abordará la raíz de los problemas sociales subyacentes que generan este tipo de contenido en línea, ni podrá eliminar completamente el contenido ofensivo de las redes sociales.